\documentclass[a4paper]{jlreq}
\usepackage[block=par,backend=biber,style=../biblatex/jpa]{biblatex}
\usepackage{etoolbox}
\usepackage{quoting}
\usepackage{url}
\quotingsetup{vskip=1\zh}
\makeatletter
\preto{\@verbatim}{\topsep=1\zh \partopsep=1\zh }
\makeatother
\newcommand{\bibentry}{\vspace{1\zh}\noindent\textbf{■ bibエントリ}}
\newenvironment{mihon}{\textbf{出力結果}\newline\begin{minipage}{\textwidth}\small\hangindent=2\zw\vspace{1\zh}}{\end{minipage}\vspace{2\zh}}
\urlstyle{same}
\addbibresource{./sample.bib}

\title{jpa:日本心理学会指定書式風\\[.5\zh]Bib{\LaTeX}スタイル}
\author{芝田征司}
\date{\today}


\begin{document}
\maketitle
\clearpage
\tableofcontents
\clearpage

\section*{はじめに}

これは,日本心理学会が発行する『執筆・投稿の手びき』の指定フォーマットにそって文献リストを作成するためのBib{\LaTeX}用スタイルファイルです。このスタイルを用いることで,日本語文献と英語文献が混在している場合でも,日本心理学会のフォーマットにそった引用および文献リストの作成が可能です。

なお,このスタイルファイルは著者(芝田)が個人的に作成したものであり,日本心理学会によるものではありません。このスタイルファイルについて日本心理学会に問い合わせたりすることのないようにしてください。

また,できる限り日本心理学会の書式にそうように作成してありますが,このスタイルファイルによる文献リストがそのまま受理されることを保証するものではありませんので,その点はご注意ください。

\section{使用環境の準備}

ここでは{\LaTeX}を使用する場合おける設定方法や使用方法について説明します。RMarkdownでPDFを作成する場合の使用方法については,RMarkdownのマニュアルなどを参照してください。

\subsection{スタイルファイルの設置}
jpaスタイルファイルは,\texttt{.bbx},\texttt{.cbx},\texttt{.dbx}の3つのファイルで構成されています。\texttt{.bbx}ファイルには文献リスト作成のための書式,\texttt{.cbx}には文章中での文献引用に関する書式が含まれており,\texttt{.dbx}はこのスタイルで使用する追加フィールドに関する情報が含まれています。jpaスタイルを使用するには,これら3つのファイルを,{\LaTeX}の管理下にあるフォルダにコピーしてください。



\subsection{文書ファイルへの読み込み}
論文などでjpaスタイルファイルを使用するには,プリアンブルに次のように指定してください。

\begin{verbatim}
  \documentclass[a4paper]{jlreq}
  \usepackage[backend=biber,style=jpa]{biblatex}
\end{verbatim}



\subsection{文献ファイルの指定}

文献リストの作成に使用する文献ファイルは,Bib{\LaTeX}形式で指定します。文書のプリアンブルに,次のように使用する\texttt{.bib}ファイルを指定するだけです。


\begin{verbatim}
  \addbibresource{文献.bib}    
\end{verbatim}

\subsection{文献リストの作成}

文書に文献リストを追加するには,文献リストを追加したいところで次のように指定します。

\begin{verbatim}
  \printbibliography[title=引用文献]  
\end{verbatim}


「\texttt{title=}」の部分は,文献リストの見出しの設定です。「\texttt{title=}」を省略した場合,文献リストの見出しは「参考文献」となります。

ここまでの設定を含む,最小限の文書ファイルは次のようになります。ここでは,文書ファイルと同じ場所に文献ファイル「文献.bib」があるものとします。

\begin{verbatim}
  \documentclass[a4paper]{jlreq}
  \usepackage[backend=biber,style=jpa]{biblatex}
  \addbibresource{文献.bib}
  \begin{document}
  こんにちは
  \printbibliography[title=引用文献]  
  \end{document}
\end{verbatim}

\section{文献の引用}
このセクションでは,本文中における文献引用コマンドの使用方法について説明します。jpaスタイルでは独自の引用コマンドを追加したりしていませんので,Bib{\LaTeX}の引用コマンドがそのまま使えます。

\subsection{本文の一部として引用する}
たとえば,「\texttt{中沢1978}」というキーをもつ文献を本文の一部として引用したい場合には,\texttt{\textbackslash{textcite}\{\}}コマンドを用いて次のように書きます。

\begin{verbatim}
  \textcite{中沢1978}によれば,これまでの研究において……
\end{verbatim}

これは次のように表示されます。

\begin{quoting}
  \small
  \textcite{中沢1978}によれば,これまでの研究において……
\end{quoting}

なお,\texttt{\textbackslash{textcite}\{\}}コマンドの引数にコンマ区切りで複数の引用キーを指定した場合,あるいは\texttt{\textbackslash{textcites}\{\}}コマンドで複数の文献を一度に引用した場合には,それぞれの文献情報の間に「と」や「および」などの区切り文字が挿入されます。たとえば,「\texttt{\textbackslash{textcite}\{齊藤2019,森川2010\}}」のようにして2つの文献を一度に引用した場合,本文中の表示は次のようになります。

\begin{quoting}
  \small
  \textcite{齊藤2019,森川2010}によれば,これまでの研究において……
\end{quoting}

また,「\texttt{\textbackslash{textcite}\{齊藤2019,森川2010,坂野1994\}}」のように3つ以上の文献を一度に引用した場合,本文中の表示は次のようになります。

\begin{quoting}
  \small
  \textcite{齊藤2019,森川2010,坂野1994}によれば,これまでの研究において……
\end{quoting}

\subsection{括弧に入れて引用する}

文末に括弧に入れて文献情報を示したい場合には,\texttt{\textbackslash{parencite}\{\}}コマンドを使用します。

\begin{verbatim}
  ……であることが示されている\parencite{藤永2013}。
\end{verbatim}

これは,次のように表示されます。

\begin{quoting}
  \small
  ……であることが示されている\parencite{藤永2013}。
\end{quoting}

複数の文献を一度に引用したい場合は,複数のキーをコンマ区切りで並べます。その際,文献の記載順序は自動的に著者名のアルファベット順にソートされます。

\begin{verbatim}
  ……ということである\parencite{坂本2013,松井2010}。
\end{verbatim}

\begin{quoting}
  \small
  ……ということである\parencite{坂本2013,松井2010}。
\end{quoting}


\subsection{括弧なしで引用する}
\texttt{\textbackslash{parencite}\{\}}コマンドでは,文献情報の前後に括弧が自動的に挿入されますが,そうしたくない場合もあるでしょう。その場合には,\texttt{\textbackslash{cite}\{\}}コマンドを使用することで,前後の括弧を非表示にできます。その場合,文献情報の前後の括弧は自分で入力する必要があります。

\begin{verbatim}
  ……ということである(\cite{坂本2013,松井2010})。
\end{verbatim}

\begin{quoting}
  \small
  ……ということである(\cite{坂本2013,松井2010})。
\end{quoting}

\subsection{その他の引用方法}
Bib{\LaTeX}では,これ以外にもさまざまな引用コマンドが用意されています。それらすべてのコマンドには対応しきれていませんが,著者名の特殊な並びや表記を目的とした引用コマンドでない限りは機能すると思いますので,いろいろと試してみてください。Bib{\LaTeX}の引用コマンドの詳細については,Bib{\LaTeX}のマニュアルを参照してください。

なお,日本心理学会の書式では,「山田 太郎」と「山田 次郎」のように,同姓の著者による文献が存在する場合で,出版年も同じである文献については,「山田太郎(2021)」と「山田次郎(2021)」のように,筆頭著者をフルネームで表示することになっていますが\footnote{英文の場合は,まずはファーストネームのイニシャルを使用し,それでも同じ場合にフルネームを使用します。},同姓の別著者による文献であっても,出版年が異なっている場合には,「山田太郎(2021)」や「山田次郎(2022)」のように書かず,「山田(2021)」や「山田(2022)」と書くことになっています。

しかしBib{\LaTeX}では著者姓の重複のみがチェックされるため,このような区別ができません。同姓別著者で出版年まで同じというケースは非常に稀だと思われますので,jpaスタイルでは,著者名の曖昧さ回避は働かないように設定してあります。もし著者名の曖昧さ回避が必要な場面に遭遇した場合には,次のように\texttt{\textbackslash{cite*}\{\}}コマンドで文献の出版年のみを表示させ,著者名については手動で書くといった形で対処する必要があります。

\begin{verbatim}
  ……ということである(山田太郎, \cite*{山田太郎2021}; 山田次郎, \cite*{山田次郎2022})。
\end{verbatim}

\begin{quoting}
  \small
  ……ということである(山田太郎, 2021; 山田次郎, 2022)。
\end{quoting}


なお,筆頭著者が同じでも,第2,第3著者まで表記すれば区別できるようなケースについては,区別可能なところまで著者名を記載することになっていますが,こちらについてはBib{\LaTeX}が問題なく自動的に処理してくれます。


\section{文献ファイルの作成}

Bib{\LaTeX}による文献リスト作成で一番重要な部分は,文献ファイル(\texttt{.bib})をしっかり作成して管理しておくことです。文献ファイルが適切な形で作成されていないと,文献リストに必要な情報を正しく記載できません。このセクションでは,jpaスタイル用に(\texttt{.bib})を作成する際の注意点について説明します。

\subsection{エントリ形式の基本}

Bib{\LaTeX}の\texttt{.bib}ファイルは,一部に仕様の変更や拡張が行われていますが,基本的にはBib{\TeX}で用いられるものと同じです。典型的には,次のような形で文献情報を入力します。

\begin{verbatim}
@article{矢嶋2013,
  author = {矢嶋, 美保 and 長谷川, 晃},
  sortname = {Yajima, Miho and Hasegawa, Akira},
  date = {2013},
  title = {家族機能が中学生の社交不安に及ぼす影響},
  subtitle = {日本の親子のデータを用いた検討},
  journal = {感情心理学研究},
  volume = {27},
  number = {3},
  pages = {83-94},
  doi = {10.4092/jsre.27.3_83},
  language = {japanese},
\end{verbatim}

\texttt{author}に著者情報,\texttt{title}に論文タイトルというように,それぞれのフィールドに対応する内容を入力していきます。日本語の文献の場合,著者名は「姓, 名」のようにコンマで区切って記載します。また,このままでは著者名のソートができませんので,\texttt{sortname}フィールドにローマ字表記で著者情報を入力します。このとき,著者名は「Yajima, Miho」と「Miho Yajima」のどちらの形式でも問題ありません。

日本語文献では,必ず\texttt{language}フィールドに\texttt{japanese}と入力してください。この情報がないと,この文献を日本語文献として適切に扱うことができません。英語文献の場合には,\texttt{sortname}や\texttt{language}は不要です\footnote{ただし,たとえば文献リストには「van Gogh」と記載したいけれども,「van」は無視してソートしたいというような場合には,外国人の名前であっても\texttt{sortname}フィールドが必要になることがあります。}。



出版年については,Bib{\TeX}では\texttt{year}というフィールドを用いるのが一般的でしたが,Bib{\LaTeX}では\texttt{date}に置き換えられています。ただし,\texttt{year}フィールドに入力されている値も認識してくれますので,すでに作成済みの\texttt{.bib}ファイルを無理に修正する必要はありません。

なお,\texttt{year}フィールドには数字以外のものを入力しても問題ありませんでしたが,\texttt{date}フィールドの内容は日付型に限定されているため,たとえばここに「印刷中」のような内容は入力できません。印刷中の論文については,\texttt{date}フィールドは省略し,次のように\texttt{pubstate}フィールドに「印刷中」と入力してください。


\begin{verbatim}
@article{長谷川99,
  author = {長谷川,龍樹 and 多田, 奏恵 and 米満, 文哉 and 池田, 鮎美 and 
            山田, 祐樹 and 高橋, 康介 and 近藤, 洋史},
  sortname = {Ryuju Hasegawa and Kanae Tada and Fumiya Yonemitsu and Ayumi Ikeda 
            and Yuki Yamada and Kohske Takahashi and Hirohito M. Kondo},
  journal = {心理学研究},
  title = {実証的研究の事前登録の現状と実践},
  subtitle = {OSF事前登録チュートリアル},
  pubstate = {印刷中},
  language = {japanese},
}
\end{verbatim}

また,全集など,出版年が複数年に渡る場合には,その出版年の最初と最後を次のようにスラッシュで区切って記載します。これにより,これが日付の範囲であることが認識され,適切に処理されるようになります。ハイフン(-)は使用できませんので注意してください。

\begin{verbatim}
@mvbook{Freud1956,
  author = {Freud, S.},
  date = {1956/1974},
  title = {Standard editions of complete psychological works of Sigmund Freud},
  volumes = {1-24},
  publisher = {Hogarth Press},
}
\end{verbatim}


\section{各文献タイプのエントリ例}
ここからは,日本心理学会『執筆・投稿の手びき』(2022年版)の各文献タイプの書式と,それに対応する\texttt{.bib}ファイルのエントリ例をみていきましょう。

\subsection{書籍}

書籍の文献情報は,英語文献,日本語文献ともに,著者名,刊行年,書籍名,版数(初版以外),出版社を記載します。英文書籍の場合,書籍名はイタリック体,版はed.で表記します。


\subsubsection{一般的な例}

一般的な書籍では,文献情報を次のように入力します。
\begin{verbatim}
@book{引用キー,
  author    = {著者名},
  sortname  = {著者名読み}, ※ 英語書籍では不要
  date      = {刊行年},
  title     = {書籍名},
  subtitle  = {サブタイトル(省略可)},
  publisher = {出版社},
}
\end{verbatim}
    
実際の入力例は次のようになります。

\bibentry
\begin{verbatim}
@book{Clement2002,
  author    = {Clement, E.},
  date      = {2002},
  title     = {Cognitive Flexibility},
  subtitle  = {The Cornerstone of Learning},
  publisher = {Wiley},
}
\end{verbatim}

\begin{mihon}
    \fullcite{Clement2002}
\end{mihon}

\bibentry
\begin{verbatim}
@book{Rosen2015,
  author    = {Rosen, L. D. and Cheever, N. and Carrier, L. M.},
  date      = {2015},
  title     = {The Wiley Blackwell Handbook of Psychology, Technology and Society},
  publisher = {Wiley},
}
\end{verbatim}


\begin{mihon}
  \nocite{Rosen2015} %<- fullciteでは著者名を全員表示できないので文献リストの内容をコピペ
  Rosen, L. D., Cheever, N., \& Carrier, L. M. (2015). \emph{The Wiley Blackwell Handbook of Psychology, Technology and Society}. Wiley
\end{mihon}
    
\bibentry
\begin{verbatim}
@book{一川2016,
  author    = {一川, 誠},
  sortname  = {Ichikawa, Makoto},
  date      = {2016},
  title     = {「時間の使い方」を科学する},
  subtitle  = {思考は10時から14時,記憶は16時から},
  publisher = {PHP研究所},
  language  = {japanese},
}
\end{verbatim}
        
\begin{mihon}
  \fullcite{一川2016}
\end{mihon}
    

\subsubsection{新・改訂版}
版の情報は\texttt{edition}フィールドに入力します。英語書籍の場合,版数は数字のみで構いません。日本語書籍の場合は,「第2版」のように入力します。

\begin{verbatim}
@book{引用キー,
  author    = {著者名},
  sortname  = {著者名読み}, ※ 英語書籍では不要
  date      = {刊行年},
  title     = {書籍名},
  subtitle  = {サブタイトル(省略可)},
  edition   = {版数},
  publisher = {出版社},
}
\end{verbatim}


\bibentry
\begin{verbatim}
@book{APA2013,
  author    = {{American Psychiatric Association}},
  date      = {2013},
  title     = {Diagnostic and statistical manual of mental disorders},
  edition   = {5}, 
  publisher = {American Psychiatric Association},
}
\end{verbatim}



\begin{mihon}
  \fullcite{APA2013}
\end{mihon}

\bibentry
\begin{verbatim}
@book{長谷川2016,
  author    = {長谷川, 寿一 and 東條, 正城 and 大島, 尚 and 丹野, 義彦 and 廣中, 直行},
  sortname  = {Toshikazu Hasegawa and Masaki Tojo and Takashi Ohshima and 
               Yoshihiko Tanno and Naoyuki Hironaka},
  date      = {2016},
  title     = {はじめて出会う心理学},
  edition   = {第3版}, 
  publisher = {有斐閣},
  language  = {japanese},
  }
\end{verbatim}
        
\begin{mihon}
  \nocite{長谷川2016}%
  長谷川 寿一・東條 正城・大島 尚・丹野 義彦・廣中 直行(2016). はじめて出会う心理学 第3版 有斐閣
\end{mihon}

\subsubsection{編集書}
英語書籍の場合,編集者はEd.(単数)またはEds.(複数)と省略表記するのがルールですが,単数と複数の区別はBib{\LaTeX}により自動的に処理されます。また,日本語の編集書では,編集者の他に監修者の名前が書かれていることがあります。その場合,監修者の名前を\texttt{author}フィールドに,編集者の名前は\texttt{editor}フィールドに入力します。さらに,\texttt{authortype}フィールドに監修者の役割(監修)を入力します。なお,その際,\texttt{sortname}フィールドには監修者名も含めて入力します。

\begin{verbatim}
@book{引用キー,
  author     = {監修者名}, ※ 監修者がいなければ省略
  authortype = {役割}, ※ 監修者がいなければ省略
  editor     = {編集者名},
  sortname   = {編著者名読み}, ※ 英語書籍では不要
  date       = {刊行年},
  title      = {書籍名},
  subtitle   = {サブタイトル(省略可)},
  edition    = {版数}, ※ なければ省略
  publisher  = {出版社},
}
\end{verbatim}

\bibentry
\begin{verbatim}
@book{Osaka2007,
  editor    = {Osaka, N. and Rentschler, I. and Biederman, I.},
  title     = {Object recognition, attention, and action},
  date      = {2007},
  publisher = {Springer},
}
\end{verbatim}

\begin{mihon}
  \nocite{Osaka2007}%
  Osaka, N., Rentschler, I., \& Biederman, I. (Eds.) (2007). \emph{Object recognition, attention, and action}. Springer.
\end{mihon}

\bibentry
\begin{verbatim}
@book{堀2009,
  author     = {堀, 洋道},
  authortype = {監修},
  sortname   = {Hori, Hiromichi and Yoshida, Fujio and Matsui, Yutaka and 
                Miyamoto, Sosuke},
  editor     = {吉田, 富二雄 and 松井, 豊 and 宮本, 聡介},
  date       = {2009},
  title      = {新編 社会心理学},
  edition    = {改訂版},
  publisher  = {福村出版},
  language   = {japanese},
}  
\end{verbatim}

\begin{mihon}
  \nocite{堀2009}%
  堀 洋道(監修) 吉田 富二雄・松井 豊・宮本 聡介(編)(2009). 新編 社会心理学 改訂版 福村出版
\end{mihon}

\subsubsection{編集書中の特定章}
編集書の中の特定の章のみを引用する場合,\texttt{@inbook}タイプを使用して,章の著者,刊行年,章のタイトル,編集者,書籍名,ページ範囲,出版社を記載します。タイトルおよび書籍タイトルには,サブタイトルも設定可能です。

\begin{verbatim}
@inbook{引用キー,
  author    = {著者名}, 
  sortname  = {著者名読み}, ※ 英語書籍では不要
  date      = {刊行年},
  title     = {章のタイトル},
  booktitle = {書籍のタイトル},
  editor    = {編集者名},
  pages     = {ページ範囲}, 
  publisher = {出版社},
}
\end{verbatim}


\bibentry
\begin{verbatim}
@inbook{Morioka2018,
  author       = {Morioka, M.},
  date         = {2018},
  title        = {On the constitution of self-experience in the psychotherapeutic 
                  dialogue},
  booktitle    = {Handbook of Dialogical Self Theory and Psychotherapy},
  booksubtitle = {Bridging Psychotherapeutic and Cultural Traditions},
  editor       = {A. Konopka and H. J. M. Hermans and M. M. Gonçalves},
  pages        = {206-219},
  publisher    = {Routledge},
}
\end{verbatim}

\begin{mihon}
  \nocite{Morioka2018}%
  Morioka, M. (2018). On the constitution of self-experience in the psychotherapeutic dialogue. In A. Konopka, H. J. M. Hermans, \& M. M. Gonçalves (Eds.), \emph{Handbook of Dialogical Self Theory and Psychotherapy: Bridging Psychotherapeutic and Cultural Traditions} (pp. 206–219). Routledge.
\end{mihon}


\bibentry
\begin{verbatim}
@inbook{内藤2018,
  author    = {内藤, 美加},
  sortname  = {Naito, Mika},
  date      = {2018},
  title     = {記憶の発達と心的時間移動},
  subtitle  = {自閉スペクトラム症の未解決課題再考},
  booktitle = {発達障害の精神病理I},
  editor    = {鈴木, 國文 and 内海, 健 and 清水, 光恵},
  pages     = {77-96},
  publisher = {星和書店},
  language  = {japanese},
}
\end{verbatim}

\begin{mihon}
  \nocite{内藤2018}%
  内藤 美加 (2018). 記憶の発達と心的時間移動------自閉スペクトラム症の未解決課題再考------ 鈴木 國文・内海 健・清水 光恵 (編) 発達障害の精神病理 I (pp. 77–96) 星和書店
\end{mihon}

\subsubsection{数巻にわたる書籍}

数巻にわたる書籍は,\texttt{mvbook}タイプを使用してエントリを作成します。刊行年が複数年にまたがる場合には,年数をスラッシュで区切って入力します。編集者の役割(監修など)がある場合は,\texttt{editortype}フィールドで指定します。また,巻数を\texttt{volumes}フィールドに入力します。

\begin{verbatim}
@mvbook{引用キー,
  author    = {著者名}, 
  sortname  = {著者名読み}, ※ 英語書籍では不要
  date      = {刊行年},
  title     = {シリーズのタイトル},
  volumes   = {巻数},
  publisher = {出版社},
}
\end{verbatim}

\bibentry
\begin{verbatim}
@mvbook{Freud1956,
  author    = {Freud, S.},
  date      = {1956/1974},
  title     = {Standard editions of complete psychological works of Sigmund Freud},
  volumes   = {1-24},
  publisher = {Hogarth Press},
}
\end{verbatim}
    
\begin{mihon}
    \fullcite{Freud1956}
\end{mihon}

\bibentry
\begin{verbatim}
@mvbook{野島2018,
  editor     = {野島, 一彦 and 繁桝, 算男},
  editortype = {監修},
  sortname   = {Nojima, Kazuhiko and Shigemasu, Kazuo},
  date       = {2018/2020},
  series     = {公認心理師の基礎と実践},
  volumes    = {全23巻},
  publisher  = {遠見書房},
  language   = {japanese}
}
\end{verbatim}
    
\begin{mihon}
  \nocite{野島2018}
  野島 一彦・繁桝 算男 (監修). (2018–2020). 公認心理師の基礎と実践 (全23巻) 遠見書房.
\end{mihon}

\subsubsection{数巻にわたる書籍の特定の1巻}

数巻にわたる書籍のうち,1巻のみを引用する場合は,エントリの作成方法には2とおりのパターンがあります。

\paragraph{従来型の方法}
従来型(Bib{\TeX}型)の方法に近いやり方は,シリーズ全体の情報とその巻の情報を\texttt{incollection}タイプのエントリとして作成するというものです。この場合,対象巻の著者・編者は\texttt{author}フィールドに,シリーズ全体の編集者は\texttt{editor}フィールドに入力します。

引用する巻の著者役割が「編集者」である場合には,\texttt{authortype}フィールドに「\texttt{editor}」(外国語文献の場合)または「\texttt{編}」(日本語文献の場合)と入力します。また,シリーズ全体の編集者の役割が,「\texttt{editor}」,「\texttt{編}」以外である場合は,\texttt{editortype}フィールドにその役割を記入します。

シリーズのタイトルは,\texttt{series}フィールドに入力します。

\begin{verbatim}
@incollection{引用キー,
  author    = {巻著者名}, 
  sortname  = {巻著者名読み}, ※ 英語書籍では不要
  date      = {刊行年},
  title     = {巻タイトル},
  volume    = {巻数}, ※省略可
  editor    = {シリーズ編集者},
  series    = {シリーズ名},
  publisher = {出版社},
}
\end{verbatim}


\bibentry
\begin{verbatim}
@incollection{Lamb2015,
  author     = {Lamb, M. E.},
  authortype = {editor},
  date       = {2015},
  title      = {Socioemotional processes},
  editor     = {R. M. Lerner},
  editortype = {Series Ed.},
  volume     = {3},
  series     = {Handbook of child psychology and developmental science},
  publisher  = {Wiley},
}
\end{verbatim}
    
\begin{mihon}
  \fullcite{Lamb2015}
\end{mihon}


\bibentry
\begin{verbatim}
@incollection{浅野2020,
  author     = {浅野, 倫子 and 横澤, 一彦},
  authortype = {編},
  sortname   = {Asano, Michiko and Yokosawa, Kazuhiko},
  date       = {2020},
  title      = {共感覚},
  subtitle   = {統合の多様性},
  editor     = {横澤, 一彦},
  editortype = {監修},
  series     = {シリーズ統合的認知},
  publisher  = {勁草書房},
  language   = {japanese},
}
\end{verbatim}
    
\begin{mihon}
    \fullcite{浅野2020}
\end{mihon}


\paragraph{\texttt{related}フィールドを用いる方法}
上記の方法とは別に,Bib{\LaTeX}の\texttt{related}フィールドを利用してエントリを作成する方法もあります。この場合,シリーズ全体の情報は\texttt{mvbook}のエントリとして作成し,その巻の情報については\texttt{book}タイプのエントリとして作成します。そして,その巻の\texttt{related}フィールドにシリーズの引用キーを,\texttt{relatedtype}に「\texttt{mvbook}」を指定します。

この場合,\texttt{\textbackslash{textcite}\{\}}などの引用コマンドには\texttt{book}で作成したその巻の引用キーを使用します。するとこの書籍が引用された際,Bib\LaTeX{}は\texttt{related}フィールドから関連書籍の情報を取り出し,それを元にしてシリーズの情報を文献リストに記載します。

\vspace{1\zh}
\textbf{シリーズ全体についてのエントリ}

\begin{verbatim}
@mvbook{引用キー,
  editor    = {シリーズ編集者名}, 
  sortname  = {編集者名読み}, ※ この文献そのものを引用しないのであれば不要
  date      = {刊行年}, ※ この文献そのものを引用しないのであれば不要
  series    = {シリーズ名},
  publisher = {出版社}, ※ この文献そのものを引用しないのであれば不要
}
\end{verbatim}

\textbf{引用する巻についてのエントリ}

\begin{verbatim}
@book{引用キー,
  editor    = {巻著者名}, 
  sortname  = {巻著者名読み}, ※ 英語書籍では不要
  date      = {刊行年},
  title     = {巻タイトル},
  publisher = {出版社},
}
\end{verbatim}


実際には,たとえば次のような形でエントリを作成します。

\bibentry
\begin{verbatim}
シリーズ全体
@mvbook{Lerner:Handbook,
  editor     = {R. M. Lerner},
  editortype = {Series Ed.},
  series     = {Handbook of Child Psychology and Developmental Science},
}

引用対象
@book{Overton2015,
  editor      = {W. F. Overton and P. C. M. Molenaar},
  publisher   = {Wiley},
  title       = {Theory and Method},
  date        = {2015},
  volume      = {1},
  related     = {Lerner:Handbook},
  relatedtype = {mvbook},
}
\end{verbatim}

このエントリを\texttt{\textbackslash{textcite}\{Overton2015\}}として引用すると,この文献情報の文献リストにおける記載は次のようになります。

\vspace{1\zh}
\begin{mihon}
  \fullcite{Overton2015}
\end{mihon}

日本語の書籍の場合も同様です。\newline

\bibentry
\begin{verbatim}
シリーズ全体
@mvbook{大山:心理学研究法,
  editor     = {大山, 正},
  editortype = {監修},
  publisher  = {誠信書房},
  series     = {心理学研究法},
  language   = {japanese},
}

引用対象の巻
@book{箱田2012,
  editor      = {箱田, 裕司},
  sortname    = {Hakoda, Yuji},
  date        = {2012},
  title       = {認知},
  publisher   = {誠信書房},
  number      = {2},
  language    = {japanese},
  related     = {大山:心理学研究法},
  relatedtype = {mvbook},
}
\end{verbatim}

このエントリを\texttt{\textbackslash{textcite}\{箱田2012\}}として引用すると,この文献情報の文献リストにおける記載は次のようになります。

\vspace{1\zh}
\begin{mihon}
  \fullcite{箱田2012}
\end{mihon}

\subsubsection{翻訳書}
文献リストにおける翻訳書の表記の仕方は,日本語訳された書籍と他の外国語から英語に翻訳された書式とで大きく異なります\footnote{翻訳書では,本文中への引用形式も日本語書籍と英語書籍とで異なります。}が,Bib\LaTeX{}を使用すれば,適切な書式の選択は自動的に行われます。

翻訳書の場合も,複数巻で構成されるシリーズの中から1冊を引用する場合と同様に,エントリの作成方法には従来型の方法と\texttt{related}フィールドを用いた方法の2とおりがあります。

\paragraph{従来型の方法}
従来型の方法では,翻訳書の情報と原著の情報を1つの\texttt{book}エントリとして作成します。その際,翻訳者の名前は\texttt{translator}フィールドに入力します。日本語訳された書籍で,監訳者と訳者が別になっている場合には,\texttt{translator}フィールドに監訳者名を入力し,\texttt{translatortype}を「\texttt{監訳}」にしてください。監訳者以外の翻訳者名は,\texttt{translatora}フィールドに入力します。

なおこの場合,\texttt{author}フィールドには,翻訳者ではなく原著者の名前を入力してください。日本語訳された書式の場合,\texttt{author}フィールドには原著者名をカタカナ表記で入力し,原語表記の原著者名は\texttt{origauthor}フィールドに入力します。

翻訳書の出版年は\texttt{date}フィールド,原書の出版年は\texttt{origdate}フィールドを使用してください。日本語訳された書籍の場合には,\texttt{origtitle}(原書タイトル)と\texttt{origpublisher}(原書の出版社)の情報も必要です。

\begin{verbatim}
@book{引用キー,
  author        = {原著者名}, ※ 日本語の翻訳書ではここはカタカナ
  origauthor    = {原著者名}, ※ 原語表記(英訳書の場合は記載不要)
  sortname      = {原著者名読み}, ※ 英訳書では不要
  origdate      = {原書出版年},
  origtitle     = {原書タイトル}, ※ 英訳書では不要
  origpublisher = {原書出版社}, ※ 英訳書では不要
  translator    = {翻訳者名},
  date          = {翻訳書出版年},
  title         = {翻訳書タイトル},
  publisher     = {出版社},
}
\end{verbatim}

実際の文献の例は次のとおりです。なお,これは英訳された書籍の例ですが,\texttt{sortname}フィールドを使用しています。これは,原著者名をvon Helmholtzの形で引用するために著者名の「von Helmholtz」の部分を\{\ \}でくくっているためです。この場合,そのままでは著者名順に文献をソートする際に「von Helmholtz」がキーワードとして用いられることになるのですが,\texttt{sortname}フィールドに\{\ \}を含まない形で著者名を記入することで,それを回避しているのです。

なお,この場合には,本文中の引用部分は「\textcite{Helmholtz1925}」の形で表示されます。

\bibentry
\begin{verbatim}
@book{Helmholtz1925,
  author         = {{von Helmholtz}, H.},
  sortname       = {von Helmholtz, H.},
  origdate       = {1910},
  translator     = {J. P. C. Southall},
  translatortype = {Ed., \& Trans.},
  date           = {1925},
  title          = {Treatise on Physiological Optics},
  volume         = {3},
  publisher      = {Optical Society of America},
}
\end{verbatim}

\vspace{1\zh}
\begin{mihon}
  \fullcite{Helmholtz1925}
\end{mihon}

日本語に翻訳された書籍の場合のエントリは次のようになります。この場合,本文中の引用部分は「\textcite{Izard1991}」のように表示されます。

\bibentry
\begin{verbatim}
@book{Izard1991,
  author         = {イザード, C. E.},
  sortname       = {Izard, C. E.},
  origauthor     = {Izard, C. E.},
  origdate       = {1991},
  origtitle      = {The psychology of emotions},
  origpublisher  = {Plenum Press},
  translator     = {荘厳, 舜哉},
  translatortype = {監訳},
  translatora    = {比較発達研究会},
  date           = {1996},
  title          = {感情心理学},
  publisher      = {ナカニシヤ出版},
  language       = {japanese},
}
\end{verbatim}

\vspace{1\zh}
\begin{mihon}
  \fullcite{Izard1991}
\end{mihon}

\paragraph{\texttt{related}フィールドを用いる方法}
翻訳書のエントリに\texttt{related}フィールドを使用する場合には,原書の情報と翻訳書の情報をそれぞれ別の\texttt{book}タイプのエントリとして作成し,翻訳書の\texttt{related}フィールドに原書の引用キーを入力します。また,これが翻訳書であることを示すために,\texttt{relatedtype}フィールドに「\texttt{translationof}」と入力してください。

翻訳書のエントリでは,\texttt{author}フィールドには原書の著者名を,翻訳者の名前は\texttt{translator}フィールドに入力します。日本語の翻訳書の場合,\texttt{author}フィールドは原書の著者名をカタカナで入力してください。なお,\texttt{related}フィールドを使用する場合には,原書著者名の原語表記は\texttt{related}フィールドを介して取得されますので,\texttt{origauthor}フィールドに別途入力したりする必要はありません。

\vspace{1\zh}
\textbf{原書のエントリ}

\begin{verbatim}
@book{引用キー,
  author    = {著者名}, 
  sortname  = {著者名読み}, ※ この文献そのものを引用しないのであれば不要
  date      = {刊行年},
  title     = {書籍タイトル}, ※ 英訳書の原書の場合には不要
  publisher = {出版社}, ※ 英訳書の原書の場合には不要
}
\end{verbatim}

\textbf{翻訳書のエントリ}

\begin{verbatim}
@book{引用キー,
  author    = {著者名}, 
  sortname  = {著者名読み}, ※ この文献そのものを引用しないのであれば不要
  date      = {刊行年},
  title     = {書籍タイトル}, ※ 英訳書の原書の場合には不要
  publisher = {出版社}, ※ 英訳書の原書の場合には不要
}
\end{verbatim}

このタイプの文献を引用する場合は,翻訳書の引用キーを使用してください。たとえば,次の例では,原書の引用キーである「Katz1930」ではなく,翻訳書の引用キー「Katz1935」を使用して引用します。その場合,本文中の引用書式は「\textcite{Katz1935}」のようになります。

\bibentry
\begin{verbatim}
原書のエントリ
@book{Katz1930,
  author      = {Katz, D.},
  date        = {1930},
}

翻訳書のエントリ
@book{Katz1935,
  author      = {Katz, D.},
  translator  = {R. B. MacLeod and C. W. Fox},
  publisher   = {Kegan Paul},
  title       = {The world of colour},
  date        = {1935},
  related     = {Katz1930},
  relatedtype = {translationof},
}
\end{verbatim}

\begin{mihon}
\fullcite{Katz1935}
\end{mihon}

また,日本語訳された書籍の場合には次のようになります。英訳された書籍の場合と違い,日本語訳された書籍の場合には原書の情報をすべて表示する必要がありますので,原書のエントリにもタイトルや出版社などの情報をすべて記入します。この場合,文献の引用には翻訳書の引用キー「ローズ2008」を使用します。その際,本文中の引用書式は「\textcite{ローズ2008}」のようになります。

\bibentry
\begin{verbatim}
原書のエントリ
@book{Rosen2005,
  author    = {Rosen, N. J.},
  date      = {2005},
  title     = {If only},
  subtitle  = {How to turn regret into opportunity},
  location  = {New York},
  publisher = {Broadway}
}

翻訳書のエントリ
@book{ローズ2008,
  author          = {ローズ, N. J.},
  sortname        = {Rosen, N. J.},
  date            = {2008},
  translator      = {村田, 光二},
  translatortype  = {監訳},
  title           = {後悔を好機に変える},
  subtitle        = {イフ・オンリーの心理学},
  publisher       = {ナカニシヤ出版},
  language        = {japanese},
  related         = {Rosen2005},
  relatedtype     = {translationof},
}
\end{verbatim}

\begin{mihon}
  \fullcite{ローズ2008}
\end{mihon}

\subsubsection{再版}
『執筆・投稿の手びき』では,このタイプは英語書籍についてのみ記載されています。このタイプの文献は,\texttt{book}タイプのエントリに\texttt{origdate}と\texttt{origpublisher}を記入する形で作成します。


\begin{verbatim}
@book{引用キー,
  author        = {著者名}, 
  date          = {刊行年},
  title         = {書籍タイトル},
  publisher     = {出版社},
  origdate      = {原書刊行年},
  origpublisher = {原書出版社},
}
\end{verbatim}
  
この場合の入力例は次のとおりです。この文献を引用する場合,本文中の引用形式は「\textcite{Adler1970}」のようになります。

\bibentry
\begin{verbatim}
@book{Adler1970,
  author        = {Adler, A.},
  date          = {1970},
  title         = {The education of children},
  publisher     = {Gateway},
  origdate      = {1930},
  origpublisher = {George Allen \& Unwin},
}
\end{verbatim}

\begin{mihon}
  \fullcite{Adler1970}
\end{mihon}

\subsubsection{自費出版}
『執筆・投稿の手びき』では,このタイプは日本語書籍についてのみ記載されています。手引きに記載されているように,このタイプは通常の書籍と同じ形式で出版社を(自費出版)とするだけですので,通常の書籍の場合と同様に\texttt{book}タイプのエントリとして作成し,出版社に(自費出版)と入力すれば大丈夫です。

\subsection{逐次刊行物}
次に,学術論文など逐次刊行物の場合について説明します。

\subsubsection{論文}
論文の文献リスト形式については,著者名の区切りが異なることなどを除けば,文字書籍の場合ほど日本語と英語で違いがありません。論文は,\texttt{article}タイプのエントリーとして作成します。

\begin{verbatim}
@article{引用キー,
  author  = {著者名}, 
  date    = {刊行年},
  title   = {表題},
  journal = {雑誌名},
  volume  = {巻数},
  number  = {号数}, *巻をとおしてページ番号が付与されている場合は省略可
  pages   = {ページ数},
  doi     = {doi}, *ない場合は省略
}
\end{verbatim}

英語論文の場合のエントリは次のようになります。

\bibentry
\begin{verbatim}
@article{Takahashi2017,
  author  = {Takahashi, N. and Isaka, Y. and Yamamoto, T. and Nakamura, T.},
  date    = {2017},
  title   = {Vocabulary and Grammar Differences Between Deaf and Hearing 
             Students},
  journal = {Journal of Deaf Studies and Deaf Education},
  volume  = {22},
  number  = {1},
  pages   = {88-104},
  doi     = {10.1093/deafed/enw055},
}
\end{verbatim}

\begin{mihon}
  \nocite{Takahashi2017}
  Takahashi, N., Isaka, Y., Yamamoto, T., \& Nakamura, T. (2017). Vocabulary and Grammar Differences Between Deaf and Hearing Students. \emph{Journal of Deaf Studies and Deaf Education, 22} (1),
88–104. \url{https://doi.org/10.1093/deafed/enw055}
\end{mihon}

\begin{mihon}
  \nocite{Takahashi2017}
\end{mihon}
日本語論文の場合も,文献エントリは同じ形です。

\bibentry
\begin{verbatim}
@article{川上2019,
  author   = {川上, 直秋},
  sortname = {Kawakami, Naoaki},
  date     = {2019},
  title    = {指先が変える単語の意味},
  subtitle = {スマートフォン使用と単語の感情価の関係},
  journal  = {心理学研究},
  volume   = {91},
  number   = {1},
  pages    = {23-33},
  doi      = {10.4992/jjpsy.91.18060},
  language = {japanese},
} 
\end{verbatim}

\begin{mihon}
  \fullcite{川上2019}
\end{mihon}

なお,2022年版の書式では,APAスタイルの第7版に合わせ,著者名を最大20人まで記載することになりました。著者が20人を超える場合は,19人まで名前を記載したのち,「\ldots」で途中を省略して最後の著者名を書くというのがルールです。
連名著者が20人を超える場合というのはそうそうないと思いますが,jpaスタイルではこの書式にそって,エントリに含まれる著者数が20を超える場合に20人目以降の著者名を自動的に「\dots」で省略して表示するようにしてあります。実際に著者名が20名を超える論文の例を見つけることができなかったので,架空の文献を例に見てみましょう。

\bibentry
\begin{verbatim}
  @article{Wiskunde2019,
  author   = {Wiskunde, B. and Arslan, M. and Fischer, P. and Nowak, L. and 
              Van den Berg, O. and Coetzee, L. and Juárez, U. and 
              Riyaziyyat, E. and Wang, C. and Zhang, I. and Li, P. and 
              Yang, R. and Kumar, B. and Xu, A. and Martinez, R. and  
              McIntosh, V. and Ibáñez, L. M. and Mäkinen, G. and Virtanen, E. 
              and Simisola Kristine and Neasa Méabh and Sebastião Moana 
              and Kamala Livia and Bruno Céleste and Kovács, A.},
  date     = {2019},
  journal  = {Journal of Improbable Mathematics},
  title    = {Indie pop rocks mathematics},
  subtitle = {Twenty One Pilots, Nicolas Bourbaki, and the empty set},
  volume          = {27},
  number   = {1},
  pages    = {1935-1968},
  doi      = {10.0000/3mp7y-537},
}
\end{verbatim}


\begin{mihon}
  \nocite{Wiskunde2019}
  Wiskunde, B., Arslan, M., Fischer, P., Nowak, L., Van den Berg, O., Coetzee, L., Juárez, U., Riyaziyyat, E., Wang, C., Zhang, I., Li, P., Yang, R., Kumar, B., Xu, A., Martinez, R., McIntosh, V., Ibáñez, L. M., Mäkinen, G., Virtanen, E., ... Kovács, A. (2019). Indie pop rocks mathematics: Twenty One Pilots, Nicolas Bourbaki, and the empty set. \emph{Journal of Improbable Mathematics, 27} (1), 1935–1968. https://doi.org/10.0000/3mp7y-537
\end{mihon}

日本語の場合についても,見ておきます。こちらも架空の文献です。

\bibentry
\begin{verbatim}
  @article{大谷2022,
  author   = {大谷, 美貴 and 岡田, あかり and 丸山, 直樹 and 松井, 奈緒美 and
              眞鍋, 久美子 and 坂本, 諒 and 山本, 淳一 and 西村, 一也 and 春日, 麻衣 
              and 清水, 真 and 阿部, 竜太 and 島田, 悠 and 谷口, 幸治 and 
              上田, 久美子 and 小林, 里恵 and 岡本, 有香 and 白井, 太郎 and 
              鈴木, 敏行 and 佐々木, 智恵美 and 山田, 徹 and 多田, 剛 and 
              中島, 奈津子 and 谷口, 昭彦 and 野田, 愛 and 高木, 由紀子},
  sortname = {Oya, Miki and Okada, Akari and Maruyama, Naoki and Matsui, Naomi 
              and Manabe, Kumiko and Sakamoto, Ryo and Yamamoto, Junichi and 
              Nishimura, Kazuya and Kasuga, Mai and Shimizu, Makoto and 
              Abe, Ryuta and Shimada, Yu and Taniguchi, Koji and Ueda, Kumiko 
              and Kobayashi, Satoe and Okamoto, Yuka and Shirai, Tarou and 
              Suzuki, Toshiyuki and Sasaki, Chiemi and Yamada, Toru and 
              Tada, Tsuyoshi and Nakajima, Natsuko and Taniguchi, Akihiko 
              and Noda, Ai and Takagi, Yukiko},
  date     = {2022},
  title    = {マジカルナンバー20},
  subtitle = {20を超えると何かが起こる},
  journal  = {架空心理学研究},
  volume   = {32},
  number   = {3},
  pages    = {80-89},
  doi      = {35.3582/fmp-2pahm},
  language = {japanese},
}
\end{verbatim}

\begin{mihon}
  \nocite{大谷2022}
  大谷 美貴・岡田 あかり・丸山 直樹・松井 奈緒美・眞鍋 久美子・坂本 諒・山本 淳一・西村 一也・ 春日 麻衣・清水 真・阿部 竜太・島田 悠・谷口 幸治・上田 久美子・小林 里恵・岡本 有香・白 井 太郎・鈴木 敏行・佐々木 智恵美 ... 高木 由紀子 (2022). マジカルナンバー 20------20を超えると何かが起こる------  架空心理学研究, \emph{32} (3), 80-89. https://doi.org/35.3582/fmp-2pahm
\end{mihon}

\subsubsection{年間に2冊以上刊行があるが巻数がない場合}
これは日本語の場合に関してのみ指定があるタイプの文献です。『投稿・執筆の手びき』では分けて書かれていますが,巻数がない場合は\texttt{volume}フィールドが空欄になるだけで,あとは通常の\texttt{article}タイプのエントリと変わりません。

\bibentry
\begin{verbatim}
  @article{齊藤2019,
  author   = {齊藤, 慈子},
  sortname = {Saito, Atsuko},
  date     = {2019},
  title    = {時に手を抜くイクメン,マーモセットのパパ},
  subtitle = {38年間のトレンド},
  journal  = {心理学ワールド},
  number   = {86},
  pages    = {25-26},
  language = {japanese},
}
\end{verbatim}

\begin{mihon}
  \nocite{齊藤2019}
  齊藤慈子(2019). 時に手を抜くイクメン,マーモセットのパパ------38年間のトレンド------ 心理学ワールド, No. 86, 25--26.

\end{mihon}

\subsubsection{年報・年鑑}
年報や年鑑,白書などは,\texttt{report}タイプのエントリとして作成します。出版社名・発行機関名は,\texttt{institution}フィールドを使用します。巻数や号数がある場合は,\texttt{volume}フィールドや\texttt{number}フィールドを追加してください。

\begin{verbatim}
@report{引用キー,
  author    = {著者名},
  date      = {刊行年},
  title     = {タイトル},
  institution = {出版社・発行機関},
}
\end{verbatim}

\bibentry
\begin{verbatim}
@report{法務総合研究所2019,
  author    = {法務総合研究所},
  sortname  = {Houmusougoukenkyujo},
  date      = {2019},
  title     = {令和元年版 犯罪白書},
  subtitle  = {平成の刑事政策},
  institution = {昭和情報プロセス},
  language  = {japanese}
}
\end{verbatim}

\begin{mihon}
  \fullcite{法務総合研究所2019}  
\end{mihon}

\subsubsection{紀要,その他}
日本語文献では,紀要について別立てで説明されていますが,これらは基本的に通常の論文の場合と同じです。\texttt{article}タイプのエントリとして作成します。

\bibentry
\begin{verbatim}
@article{中道2019,
  author    = {中道, 圭人},
  sortname  = {Nakamichi, Keito},
  date      = {2019},
  title     = {幼児における他者の感情推測のための表情と身体的手がかりの利用},
  journal   = {千葉大学教育学部研究紀要},
  volume    = {67},
  pages     = {285-292},
  language  = {japanese},
}
\end{verbatim}

\begin{mihon}
  \nocite{中道2019}
  中道 圭人 (2019). 幼児における他者の感情推測のための表情と身体的手がかりの利用 千葉大学教育学部研究紀要, \emph{67}, 285--292.
\end{mihon}

\subsection{オンライン資料}

ここでは,オンライン資料を引用する際のエントリ作成方法について見ていきます。

\subsubsection{刊行された冊子体がある場合}
この場合のエントリは,通常の学術論文と同じです。\texttt{article}タイプの文献としてエントリを作成します。

\subsubsection{オンライン早期公開}
冊子体が刊行される前にオンラインで早期公開されているものについては,\texttt{article}タイプの文献としてエントリを作成し,\texttt{howpublished}フィールドに「Advance online publication」と入力します。また,\texttt{doi}フィールドにdoiを記載するのを忘れないでください。

\begin{verbatim}
@article{引用キー,
  author       = {著者名}, 
  date         = {刊行年},
  title        = {表題},
  journal      = {雑誌名},
  volume       = {巻数},
  number       = {号数}, *巻をとおしてページ番号が付与されている場合は省略可
  pages        = {ページ数},
  howpublished = {Advance online publication},
  doi          = {doi},
}
\end{verbatim}

\bibentry
\begin{verbatim}
@article{Yokoyama2020,
  author       = {Yokoyama, T. and Kato, R. and Inoue, K. and Takeda, Y.},
  date         = {2020},
  title        = {Cuing Effects by Biologically and Behaviorally Relevant Symbolic 
                  Cues},
  journal      = {Japanese Psychological Research},
  doi          = {10.1111/jpr.12318},
  howpublished = {Advance online publication}
}
\end{verbatim}

\begin{mihon}
  \fullcite{Yokoyama2020}
\end{mihon}

\bibentry
\begin{verbatim}
@article{金政2021,
  author       = {金政, 祐司 and 古村, 健太郎 and 浅野, 良輔 and 荒井, 崇史},
  sortname     = {Yuji Kanemasa and Kentaro Komura and Ryosuke Asano and 
                  Takashi Arai},
  date         = {2021},
  title        = {愛着不安は親密な関係内の暴力の先行要因となり得るのか?},
  subtitle     = {恋愛関係と夫婦関係の縦断調査から},
  journal      = {心理学研究},
  doi          = {10.4992/jjpsy.92.20013},
  howpublished = {Advance online publication},
  language     = {japanese},
}

\end{verbatim}

\begin{mihon}
  \nocite{金政2021}
  金政 祐司・古村 健太郎・浅野 良輔・荒井 崇史 (2021). 愛着不安は親密な関係内の暴力の先行要因となり得るのか?------恋愛関係と夫婦関係の縦断調査から------ 心理学研究 Advance online publication. https://doi.org/10.4992/jjpsy.92.20013
\end{mihon}


\subsubsection{プレプリントされている文献}
プレプリントの文献を引用する場合は,\texttt{online}タイプのエントリを作成します。\texttt{online}タイプでは,\texttt{eprinttype}フィールドにアップロードサイト名を記載します。

\begin{verbatim}
@online{引用キー,
  author     = {著者名}, 
  date       = {刊行年},
  title      = {表題},
  eprinttype = {アップロードサイト名},
  doi        = {doi},
}
\end{verbatim}
  
\bibentry
\begin{verbatim}
@online{Yoshimura2021,
  author     = {Yoshimura, N. and Morimoto, K. and Murai, M. and Kihara, Y. and
                Marmolejo-Ramos, F. and Kubik, V. and Yamada, Y.},
  date       = {2021},
  title      = {Age of smile},
  subtitle   = {A cross-cultural replication report of Ganel and 
                Goodale (2018)},
  eprinttype = {PsyArXiv},
  doi        = {10.31234/osf.io/dtx6}
}
\end{verbatim}

\begin{mihon}
  \nocite{Yoshimura2021}
  Yoshimura, N., Morimoto, K., Murai, M., Kihara, Y., Marmolejo-Ramos, F., Kubik, V., \& Yamada, Y. (2021). \emph{Age of smile: A cross-cultural replication report of Ganel and Goodale (2018)}. PsyArXiv. https://doi.org/10.31234/osf.io/dtx6
\end{mihon}

\subsubsection{オンラインでのみ閲覧可能でdoiがある場合}
オンラインでしか閲覧できない資料でdoiがあるものについては,通常の論文と同様に\texttt{article}タイプのエントリとして作成します。その際,必ず\texttt{doi}フィールドにdoiを記載してください。

\bibentry
\begin{verbatim}
@article{Katahira2020,
  author  = {Katahira, K. and Kunisato, Y. and Yamashita, Y. and Suzuki, S.},
  date    = {2020},
  title   = {Commentary: ``A robust data-driven approach identifies four  
             personality types across four large data sets.''},
  journal = {Frontiers in Big Data},
  volume  = {3},
  pages   = {8},
  doi     = {10.3389/fdata.2020.00008},
} 
\end{verbatim}

\begin{mihon}
  \nocite{Katahira2020}
  Katahira, K., Kunisato, Y., Yamashita, Y., \& Suzuki, S. (2020). Commentary: ``A robust data-driven approach identifies four personality types across four large data sets.'' \emph{Frontiers in Big Data, 3}, 8. https://doi.org/10.3389/fdata.2020.00008
\end{mihon}

\bibentry
\begin{verbatim}
  @article{矢嶋2013,
  author   = {矢嶋, 美保 and 長谷川, 晃},
  sortname = {Yajima, Miho and Hasegawa, Akira},
  date     = {2013},
  title    = {家族機能が中学生の社交不安に及ぼす影響},
  subtitle = {日本の親子のデータを用いた検討},
  journal  = {感情心理学研究},
  volume   = {27},
  number   = {3},
  pages    = {83-94},
  doi      = {10.4092/jsre.27.3_83},
  language = {japanese},
} 
\end{verbatim}

\begin{mihon}
  \fullcite{矢嶋2013}
\end{mihon}

\subsubsection{オンラインでのみ閲覧可能で,doiがない場合}

オンラインのみの資料でdoiがないものについては,\texttt{online}タイプのエントリとして作成し,\texttt{organization}フィールドにウェブサイト名を記載します。また,その資料のURLを\texttt{url}フィールドに,アクセス日時を\texttt{urldate}フィールドに記載します。\texttt{urldate}は\texttt{date}型の入力しか受け付けませんので,英語文献の場合も日本語文献の場合も\texttt{year-month-date}の形式で入力してください。

\begin{verbatim}
@online{引用キー,
  author       = {著者名}, 
  date         = {刊行年},
  title        = {表題},
  organization = {ウェブサイト名},
  url          = {資料のURL},
  urldate      = {アクセス日時},
}
\end{verbatim}
   
\bibentry
\begin{verbatim}
@online{Abrams2020,
  author       = {Abrams, Z.},
  date         = {2020},
  title        = {Building a safe space in the pandemic},
  organization = {American Psychological Association},
  url          = {https://www.apa.org/topics/covid-19/pandemic-safe-space#},
  urldate      = {2021-12-31}
}
\end{verbatim}

\begin{mihon}
  \fullcite{Abrams2020}
\end{mihon}

\bibentry
\begin{verbatim}
@online{日本心理学会2022,
  author       = {日本心理学会},
  sortname     = {Nihonshinrigakkai},
  date         = {2022},
  title        = {執筆・投稿の手びき2022年版},
  organization = {日本心理学会},
  url          = {https://psych.or.jp/manual/},
  urldate      = {2022-10-25},
  language     = {japanese},
}
\end{verbatim}

\begin{mihon}
  \fullcite{日本心理学会2022}
\end{mihon}

\subsection{その他}
\subsubsection{学位論文}
博士論文や修士論文など,学位論文は\texttt{thesis}タイプのエントリとして作成します。このタイプのエントリでは,\texttt{institution}フィールドに学位授与機関(大学)の名称,\texttt{type}フィールドに論文種別(修士論文,博士論文など)を記載します。

\begin{verbatim}
@thesis{引用キー,
  author      = {著者名}, 
  date        = {刊行年},
  title       = {表題},
  institution = {学位授与機関(大学)の名称},
  type        = {論文種別},
}
\end{verbatim}
    
\bibentry
\begin{verbatim}
@thesis{Tsukamoto2015,
  author      = {Tsukamoto, S.},
  date        = {2015},
  title       = {The Role of Psychological Essentialism in Intergroup Attitude 
                 Formation},
  institution = {Kyoto University},
  type        = {Unpublished master's thesis}
}
\end{verbatim}

\begin{mihon}
  \fullcite{Tsukamoto2015}.
\end{mihon}

\bibentry
\begin{verbatim}
@thesis{向田2009,
  author      = {向田, 久美子},
  sortname    = {Mukaida, Kumiko},
  date        = {2009},
  title       = {語りに見るライフ・スクリプトの文化心理学的研究},
  subtitle    = {文化圏間比較と世代間比較を通して},
  institution = {白百合女子大学大学院},
  type        = {博士論文},
  language    = {japanese},
}
\end{verbatim}

\begin{mihon}
  \fullcite{向田2009}
\end{mihon}

\subsubsection{学会発表}
学会発表は,\texttt{inproceedings}タイプのエントリとして作成します。予稿集や抄録などがある場合は,その名称を\texttt{booktitle}フィールドに,掲載ページ数を\texttt{pages}フィールドに記載します。それ以外の場合には,大会名を\texttt{eventtitle}フィールドに,開催地名を\texttt{location}フィールドに,発表形式を\texttt{type}フィールドに記載します。

\vspace{1\zh}
\noindent\textbf{予稿集などがある場合}
\begin{verbatim}
@inproceedings{引用キー,
  author      = {著者名}, 
  date        = {刊行年},
  title       = {表題},
  booktitle   = {ウェブサイト名},
  pages       = {資料のURL},
}
\end{verbatim}

\noindent\textbf{予稿集などがない場合}
\begin{verbatim}
@inproceedings{引用キー,
  author     = {著者名}, 
  date       = {発表年},
  title      = {表題},
  type       = {発表形式},
  eventtitle = {大会名}
  location   = {開催地名},
}
\end{verbatim}

\bibentry
\begin{verbatim}
@inproceedings{Oe2016,
  author     = {Oe, T. and Aoki, R. and Numazaki, M.},
  date       = {2016},
  title      = {Perceived causal attributions of body temperature increase
                as a moderator of the effects of physical warmth on implicit 
                associations of social warmth},
  type       = {Poster presentation},
  eventtitle = {The 17th Annual Meeting of the Society for Personality and Social 
                Psychology},
  location   = {San Diego, CA.}
}
\end{verbatim}

\begin{mihon}
  \nocite{Oe2016}
  Oe, T., Aoki, R., \& Numazaki, M. (2016). \emph{Perceived causal attributions of body temperature increase as a moderator of the effects of physical warmth on implicit associations of social warmth} [Poster presentation]. The 17th Annual Meeting of the Society for Personality and Social Psychology, San Diego, CA.
\end{mihon}

\bibentry
\begin{verbatim}
@inproceedings{都築2018,
  author    = {都築, 誉史 and 武田, 裕司 and 千葉, 元気},
  sortname  = {Tsuzuki, Takashi and Takea, Yuji and Chiba, Genki},
  date      = {2018},
  title     = {認知資源が多肢選択意思決定における魅力効果に及ぼす影響},
  subtitle  = {聴覚プローブ法を用いた実験的検討},
  booktitle = {日本心理学会第82回大会発表論文集},
  pages     = {493},
  language  = {japanese}
}
\end{verbatim}

\begin{mihon}
  \nocite{都築2018} 
  都築 誉史・武田 裕司・千葉 元気 (2018). 認知資源が多肢選択意思決定における魅力効果に及ぼす影響------聴覚プローブ法を用いた実験的検討------ 日本心理学会第82回大会発表論文集, 493.
\end{mihon}

\subsubsection{印刷中の論文}
掲載されることは確定しているが未刊行の論文については,\texttt{article}タイプエントリの\texttt{date}フィールドを空欄にし,\texttt{pubstate}フィールドに「印刷中」など,現在の状態を記載します。

\begin{verbatim}
@article{引用キー,
  author   = {著者名}, 
  title    = {表題},
  journal  = {雑誌名}
  pubstate = {状態},
}
\end{verbatim}

\bibentry
\begin{verbatim}
@article{OSeaghdha99,
  author   = {O'Seaghdha, P. G.},
  title    = {Across the great divide},
  subtitle = {Proximate units at the lexical-phonological interface},
  journal  = {Japanese Psychological Research},
  pubstate = {in press},
}
\end{verbatim}

\begin{mihon}
  \fullcite{OSeaghdha99}.
\end{mihon}

\bibentry
\begin{verbatim}
@article{長谷川99,
  author   = {長谷川,龍樹 and 多田, 奏恵 and 米満, 文哉 and 池田, 鮎美 and 
              山田, 祐樹 and 高橋, 康介 and 近藤, 洋史},
  sortname = {Ryuju Hasegawa and Kanae Tada and Fumiya Yonemitsu and Ayumi Ikeda 
              and Yuki Yamada and Kohske Takahashi and Hirohito M. Kondo},
  journal  = {心理学研究},
  title    = {実証的研究の事前登録の現状と実践},
  subtitle = {OSF事前登録チュートリアル},
  pubstate = {印刷中},
  language = {japanese},
} 
\end{verbatim}

\begin{mihon}
  \nocite{長谷川99}
  長谷川 龍樹・多田 奏恵・米満 文哉・池田 鮎美・山田 祐樹・高橋 康介・近藤 洋史 (印刷中). 実証
的研究の事前登録の現状と実践------OSF事前登録チュートリアル------ 心理学研究
\end{mihon}

\subsubsection{新聞記事および雑誌記事の引用}
新聞記事などを引用する場合は,\texttt{misc}(その他)タイプとしてエントリを作成します。発行日については,朝刊・夕刊などの区別を含め,\texttt{howpublished}フィールドに記載します。

\begin{verbatim}
@misc{引用キー,
  author       = {著者名}, 
  date         = {発行年},
  title        = {資料表題},
  journaltitle = {掲載紙(誌)名},
  howpublished = {発行日・形態},
  pages        = {掲載ページ}, 
}
\end{verbatim}

\bibentry
\begin{verbatim}
@misc{サトウ2013,
  author       = {サトウ, タツヤ},
  sortname     = {Sato, Tatsuya},
  date         = {2013},
  title        = {ちょっとココロ学},
  subtitle     = {悩み事 どうやって打開?},
  journaltitle = {読売新聞},
  howpublished = {7月8日夕刊},
  pages        = {7},
  language     = {japanese},
} 
\end{verbatim}
\begin{mihon}
  \fullcite{サトウ2013}
\end{mihon}

\clearpage
\printbibliography[title=引用文献]
\end{document}

